\documentclass[11pt]{article}
\usepackage{epsfig}
\usepackage{amssymb,amsfonts,url}
\urlstyle{rm}   
\usepackage{color}
\usepackage[utf8]{inputenc}
\usepackage{wrapfig}
\usepackage{graphicx}
\newcommand{\junk}[1]{}
\usepackage{fullpage}

\date{}
\author{}

\pagenumbering{empty}

\begin{document}

\noindent
\large {\bf popSTR: population-scale detection of STR variants} 


\normalsize 


\noindent \paragraph{Keywords:} Bioinformatics, Mathematical modeling, Motif finding, Sequence analysis,
Statistics, Algorithms

\noindent \paragraph{Abstract:} 

Microsatellites, also known as short tandem repeats (STRs), are tracts of repetitive
DNA sequences containing motifs ranging from 2-6 bases. The human reference genome contains
approximately 1 million microsatellites, covering almost 1\% of the genome (Gymrek et al., 2016).
Microsatellite analysis has a wide range of applications, including medical genetics, forensics and
construction of genetic genealogy. However, microsatellite variations are rarely considered in whole-genome
 sequencing studies, in large due to a lack of tools capable of analyzing them (Duitama et al., 2014).

 Here we present a microsatellite genotyper which is both faster and more accurate than other
methods previously presented. There are two main ingredients to our improvements. First we reduce
the amount of sequencing data necessary for creating microsatellite profiles by using previously aligned
sequencing data. Second, we use population information to train microsatellite and individual specific
error profiles. By comparing our genotyping results to genotypes generated by capillary electrophoresis
we show that our error rates are 50\% lower than those of lobSTR, another program specifically developed
to determine microsatellite genotypes.

Availability: Source code is available on Github: \url{https://github.com/snaedis88/popSTR.git}

\noindent \paragraph{Authors:} 

\noindent \paragraph{} 

{
\centering
\resizebox{\textwidth}{!}{%
\begin{tabular}{|l|l|l|l|l|l|}
\hline
\textbf{first name} & \textbf{last name} & \textbf{email}                     & \textbf{country} & \textbf{organization}                                                                      & \textbf{corresponding?} \\ \hline
Snædís              & Kristmundsdóttir   & snaedis.kristmundsdottir@decode.is & Iceland          & deCODE Genetics / Amgen                                                                    &    $\checkmark$                     \\ \hline
Brynja D.           & Sigurpálsdóttir    &                                    & Iceland          & Rejkyavik University                                                                       &                         \\ \hline
Birte               & Kehr               &                                    & Iceland          & deCODE Genetics / Amgen                                                                    &                         \\ \hline
Bjarni V.           & Halldórsson        & bjarni.halldorsson@decode.is       & Iceland          & \begin{tabular}[c]{@{}l@{}}deCODE Genetics / Amgen\\ and Rejkyavik University\end{tabular} & $\checkmark$                        \\ \hline
\end{tabular}%
}
}

\end{document}
