\noindent
\large {\bf NanoSim: nanopore sequence read simulator based on statistical characterization} 


\normalsize 


\noindent \paragraph{Keywords:} 

\noindent \paragraph{Abstract:} 

The   MinION   sequencing   platform   from   Oxford   Nanopore   Technologies   (ONT)   is   still   a  
pre-commercial  technology,  yet  it  is  generating  substantial  excitement  in  the  field  for  its  
features  –  longer  read  lengths  and  single-molecule  sequencing  in  particular.  As  groups  
start   developing   bioinformatics   tools   for   this   new   platform,   a   method   to   model   and  
simulate   the   properties   of   the   sequencing   data   will   be   valuable   to   test   alternative  
approaches  and  to  establish  performance  metrics.  Here,  we  introduce  NanoSim,  a  fast  
and   lightweight   read   simulator   that   captures   the   technology-specific   characteristics   of  
ONT  data  with  robust  statistical  models.  

The   first   step   of   NanoSim   is   read   characterization,   which   provides   a   comprehensive  
alignment-­based   analysis,   and   generates   a   set   of   read   profiles   serving   as   the   input   to  
the   next   step,   the   simulation   stage.   The   simulation   tool   uses   the   model   built   in   the  
previous  step  to  produce  in  silico  reads  for  a  given  reference  genome.  NanoSim  is  built  
on   our   observation   that   patterns   of   correct   base   calls   and   errors   (mismatches   and  
indels)  can  be  described  by  statistical  mixture  models.  Further,  the  structures  of  these  
models   are   consistent   across   chemistries   and   organisms   ({\it E.   coli}   and   {\it S.   cerevisiae}).  
NanoSim  generates  synthetic  ONT  reads  with  empirical  profiles  derived  from  reference  
datasets,   or   using   runtime   parameters.   Empirical   profiles   include   read   lengths   and  
alignment  fractions  (the  ratio  of  alignment  lengths  after  unaligned  portions  of  reads  are  
soft-­clipped   from   their   flanks   to   read   lengths).   The   lengths   of   intervals   between   errors  
(stretches   of   correct   bases)   and   error   types   are   modeled   by   Markov   chains,   and   the  
lengths  of  errors  are  drawn  from  mixed  statistical  models.  


In   this   work,   we   demonstrate   the   performance   of   NanoSim   on   publicly   available  
datasets   generated   using   R7   and   R7.3   chemistries   and   different   sequencing   kits.  
NanoSim   mimics   ONT   reads   well,   true   to   the   major   features   of   the   emerging   ONT  
sequencing  platform,  in  terms  of  read  length  and  error  modes.  The  independent  profiling  
module   grants   users   the   freedom   to   characterize   their   own   ONT   datasets,   which   is  
expected   to   perform   consistently   upon   the   improvement   of   nanopore   sequencing  
technology,  as  the  shapes  of  the  error  models  hold  among  different  datasets.  NanoSim  will   immediately   benefit   the   development   of   scalable   NGS   technologies   for   the   long  
nanopore   reads,   including   genome   assembly,   mutation   detection,   and   even  
metagenomic  analysis  software.  The  scalability  of  NanoSim  to  human-­size  genome  will  
benefit   the   development   of   scalable   NGS   technologies   for   long   nanopore   reads.  
Moreover,   a   mixture   of   in   silico   genomes   simulating   a   microbiome   will   be   helpful   for  
benchmarking   algorithms   with   applica-­tion   in   metagenomics,   including   functional   gene  
prediction,   species   detection,   comparative   metagenomics,   clinical   diagnosis.   As   such,  
we  expect  NanoSim  to  have  an  enabling  role  in  the  field. 

\noindent \paragraph{Authors:} 

