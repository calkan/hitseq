\noindent
\large {\bf Bayesian latent variable models for single-cell trajectory learning} 


\normalsize 


\noindent \paragraph{Keywords:} 

\noindent \paragraph{Abstract:} 

The transcriptomes of single cells undergoing diverse biological processes - such as differentiation or
apoptosis - display remarkable heterogeneity that is averaged over in bulk sequencing. Single-cell
sequencing itself offers only a snapshot of these processes by capturing cells of variable and unknown
progression through them. Consequently, one outstanding problem in single-cell genomics is to find
an ordering of cells (known as their pseudotime) that best reflects their progression, for which several
computational methods have been proposed.

To date, the vast majority of such methods emphasise transcriptome-wide ``data-driven'' approaches
that assume no prior knowledge of gene dynamics along the trajectory during inference. The suitability
of the inferred trajectory is typically assessed by post-hoc examination of a set of marker genes to
ensure the inferred behaviour aligns with prior assumptions. Furthermore, most current methods
are algorithmic and rely on heuristics as opposed to probabilistic models, which in the context of
bifurcations requires the pseudotimes to be first inferred prior to the identification of any bifurcation
events.


Here we introduce a general probabilistic framework for single-cell trajectory learning based on Bayesian
non-linear factor analysis and apply it to two outstanding problems in single-cell analysis. Firstly,
we demonstrate how such a framework may be used to integrate prior knowledge of gene behaviour
in trajectory inference. By assuming a parametric form of gene expression evolution across pseudo-
time we can place informative priors on parameters that govern gene behaviour within a Bayesian
statistical framework. Consequently, we remove the need for subjective post-inference checks and si-
multaneously solve related problems such as trajectory orientation and setting implicit length scales.
We demonstrate how using such methods only a small panel of marker genes are required to achieve
comparable results to transcriptome wide ‘data-driven’ alternatives. We further demonstrate how
such a method can be used to recover trajectories corresponding to known pathways in the presence
of heavily confounding effects.

The second application of our framework is to modelling bifurcations in single-cell data. By
considering a Bayesian mixture of factor analysers we simultaneously infer both the pseudotimes and
branching behaviour of the cells, which is unique compared to existing methods. We derive a Gibbs
sampler that allows for fast inference across hundreds of cells while accounting for the zero inflation
that is pertinent to single-cell RNA-seq data. Notably, by using a Bayesian framework we can integrate
prior knowledge of branch-specific gene behaviour allowing for robust inference on challenging datasets.

We introduce a flexible Bayesian framework that solves several outstanding issues in single-cell trajec-
tory learning. This framework uniquely provides a principled method for integrating prior knowledge
of gene behaviour along single-cell trajectories and allows for such trajectories to be learned from a
1small panel of marker genes. We also introduce the first statistical method for bifurcation inference
that simultaneously infers both the pseudotimes of the cells as well as the bifurcation events, pro-
viding robust trajectories as well as full uncertainty estimates. We apply our methods to a range of
both synthetic and real data, and more generally discuss the challenges of single-cell latent variable
modelling including the connection of principal component analysis to both pseudotime inference and
dropout rate. We conclude by motivating why such methods can be applied to a wide range of ‘omics’
data including modelling cancer progression and patient treatment outcomes.

\noindent \paragraph{Authors:} 

