\noindent
\large {\bf RNA Bloom: de novo RNAseq assembly with Bloom filters} 


\normalsize 


\noindent \paragraph{Keywords:} 

\noindent \paragraph{Abstract:} 

RNA-seq  is  primarily  used  in  measuring  gene  expression,  quantification  of  transcript 
abundance,  and   building  reference  transcriptomes.  Without  bias  from  a  reference  sequence,  ​
de 
novo  RNA-seq  assembly  is  particularly   useful  for  building  new  reference  transcriptomes, 
detecting  fusion  genes,  and  discovering  novel  transcripts.  A  number  of  approaches  for  de  novo 
RNA-seq  assembly  were  developed  over  the  past  six  years,  including  Trans-ABySS,  Trinity, 
Oases,  IDBA-tran,  and  SOAPdenovo-Trans.  Using  12  CPUs,  it  takes  approximately  a  day  to 
assemble  a  human  RNA-seq  sample  and  require  up  to  100GB  of  memory.  While  the  high 
memory  usage  may  be  alleviated  by  distributed  computing,  access  to  a  high-performance 
computing environment is a strict requirement for RNA-seq assembly. 

Here,  we  present  a  novel  de  novo  RNA-seq  assembler,  ``RNA-Bloom'',  that  utilizes 
Bloom  filter-based  data  structures  for  compact  storage  of  k-mer  counts  and  the  de  Bruijn  graph 
of  two  k-mer  sizes  in  memory.  Compared  to  existing  approaches,  RNA-Bloom  can  assemble  a 
human  RNA-seq  sample  with  comparable  accuracy  using  merely  10GB  of  memory,  which  is 
readily  available  on  modern  desktop  computers.  The  de  Bruijn  graph  of  two  k-mer  sizes  allows 
RNA-Bloom  to  effectively   assemble  both  lowly  and  highly  expressed  transcripts.  In  addition, 
RNA-Bloom  can assemble and quantify transcript isoforms without alignment of sequence reads, 
thus resulting in a quicker run-time than existing alignment-based protocols.  
 

\noindent \paragraph{Authors:} 

