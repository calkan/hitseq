\noindent
\large {\bf ntHash: recursive nucleotide hashing} 


\normalsize 


\noindent \paragraph{Keywords:} 

\noindent \paragraph{Abstract:} 

In bioinformatics, there are many applications that rely on cataloguing or counting DNA/RNA sequences
for indexing, querying, and similarity search. These include sequence alignment, genome and transcriptome
assembly, RNA-seq expression quantification, and error correction. An efficient way of performing such
operations is through the use of hash-based data structures, such as hash tables or Bloom filters. Therefore,
improving the performance of hashing algorithms would have a broad impact for a wide range of
bioinformatics tools.

Here, we present ntHash, a fast hash method for computing hash values for all possible sub-sequences of
length $k$ ($k$-mers) in a DNA sequence. The algorithm calculates hash values for consecutive $k$-mers in a
given sequence using a recursive approach, in which the hash value of the current $k$-mer is derived from the
hash value of the previous $k$-mer. In this work, we have implemented a cyclic polynomial rolling hash
function, and adapted it to nucleotide hashing. Particularly, we made use the reduced alphabet of DNA
sequences, and handled the reverse complementation efficiently. The proposed method also provides a fast
way for calculating multiple hash values for a given $k$-mer without repeating the whole hashing procedure
for each value. This functionality would be very useful for certain bioinformatics applications, such as
those that utilize the Bloom filter data structure.

Experimental results demonstrate substantial speed improvement over conventional approaches, while
retaining near-ideal hash value distribution. Comparison of run time of proposed method with the state-of-
the-art general-purpose hash functions demonstrates that ntHash performs over 20x faster than the closest
competitor, {\it cityhash}, the leading algorithm developed by Google.


\noindent \paragraph{Authors:} 

