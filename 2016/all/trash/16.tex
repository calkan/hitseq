\noindent
\large {\bf Evaluation of strategies for somatic mutation discovery in tumor specimens without matched germline: effect of tumor content, sequencing depth, and copy number alterations} 


\normalsize 


\noindent \paragraph{Keywords:} 

\noindent \paragraph{Abstract:} 

Large-scale multiplexed identification of somatic alterations in cancer has become
feasible with next generation sequencing (NGS). By definition, somatic alterations are
those that are found in the tumor and not the germline sequence, so the standard
approach to somatic variant detection involves comparing the tumor sequence to the
germline sequence of the same individual. However, in some situations, such as with
archival samples, blood or other constitutional tissue samples are not available to obtain
germline sequence. In order to identify somatic variants in such tumor samples, the
tumor is typically compared to a reference sample, and then the variants that are found
public germline variant databases are filtered out. However, all individuals will have
some private germline variants not found in any database. Differences in allele
frequencies between somatic and germline variants in impure tumors can also help to
differentiate somatic and germline variants. Here we will examine the extent to which
leveraging allele frequencies can help to overcome false positives due to private
germline variants in tumor only calling.

We developed a Bayesian framework to integrate the population frequency and allele
frequency information. At each position, we determine the prior probability of a germline
or somatic based on 1000 Genomes or COSMIC frequencies, respectively. We also
estimate copy number, minor allele copy number, and clonal sample fraction in order to
calculate expected allele frequencies of somatic and germline variants at each position.
As expected, the higher the clonal sample fraction, the closer the expected allele
frequencies are for somatic and germline variants. We also find that there also other
combinations of tumor content and copy number state where the expected allele
frequencies of somatic and germline variants are very similar.

Applying this framework to simulated data, we estimate coverage required for different
tumor content and copy number states. For example, to detect about 90\% of the
somatic variants in a diploid region of a 50\% tumor sample, we would only need 200X
mean target coverage, but we would need 800X mean target coverage to achieve the
same sensitivity in a 75\% tumor sample, or 1600X for an 85\% tumor sample. We then
apply the framework to a set of nine cancer samples. We find that the observed
sensitivity correlates well with the expected sensitivity based on the coverage, the clonal
sample fractions, and the copy number alterations. In silico dilutions and downsampling
experiments also confirm the expected relationships between coverage, tumor content,
and sensitivity.

We find that the Bayesian tumor only caller is able to greatly reduce false positives due
to private germline variants, with greater than 95\% of true private germline variants
correctly classified as germline. The calling precision is also significantly improved with
Bayesian approach, which has an average positive predictive value of greater than 70\%
compared to 35\% with database filtering alone. Overall the accuracy of the Bayesian
tumor only caller is greater than 99.9\%/


Our Bayesian tumor only calling approach can eliminate most false positives due to
private germline variants. However, the sensitivity of the approach is dependent tumor
content, coverage, and copy number alterations. The data presented here can be used
to design tumor only sequencing experiments with appropriate coverage based on the
sample characteristics.

\noindent \paragraph{Authors:} 

