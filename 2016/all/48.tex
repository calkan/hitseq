\noindent
\large {\bf TwoPaCo: An efficient algorithm to build the compacted de Bruijn graph from many complete genomes} 


\normalsize 


\noindent \paragraph{Keywords:} 
Algorithms, graph theory, comparative genomics, parallel computing, de Bruijn graph

\noindent \paragraph{Abstract:} 

De Bruijn graphs have been proposed as a data structure to facilitate the analysis of related
whole genome sequences, in both a population and comparative genomic settings. However, current
approaches do not scale well to many genomes of large size (such as mammalian genomes).

In this paper, we present TwoPaCo, a simple and scalable low memory algorithm for the direct
construction of the compacted de Bruijn graph from a set of complete genomes. We demonstrate that it
can construct the graph for 100 simulated human genomes in less then a day and eight real primates in
less than two hours, on a typical shared-memory machine. We believe that this progress will enable novel
biological analyses of hundreds of mammalian-sized genomes.

Availability: Our code and data is available for download from \url{github.com/medvedevgroup/TwoPaCo}

\noindent \paragraph{Authors:} 

\noindent \paragraph{} 
% Please add the following required packages to your document preamble:
% \usepackage{graphicx}
% \usepackage[normalem]{ulem}
% \useunder{\uline}{\ul}{}
{
\centering
\resizebox{\textwidth}{!}{%
\begin{tabular}{|l|l|l|l|l|l|}
\hline
\textbf{first name} & \textbf{last name} & \textbf{email}       & \textbf{country} & \textbf{organization}                 & \textbf{corresponding?} \\ \hline
Ilia                & Minkin             & ium125@psu.edu       & United States    & Pennsylvania State University         &                         \\ \hline
Son                 & Pham               &                      & United States    & Salk Institute for Biological Studies &                         \\ \hline
Paul                & Medvedev           & pashadag@cse.psu.edu & United States    & Pennsylvania State University         &  $\checkmark$                       \\ \hline
\end{tabular}%
}
}

