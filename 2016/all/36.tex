\noindent
\large {\bf Accurate Modeling and Correction of GC Content and Gene Length Bias from RNA-seq Data} 


\normalsize 


\noindent \paragraph{Keywords:} 
Data analysis, Gene expression, Next-generation sequencing

\noindent \paragraph{Abstract:} 

Several approaches exist that deal with GC content and gene length dependent biases
in RNA-seq measurements of transcript abundances. However, when computing correction factors,
existing approaches do not consider the potential interdependence and interaction of both effects.
Additionally, they do not deal satisfactorily with genes that have partially or entirely zero counts.
Thirdly, the biases may affect a large fraction of genes such that the assumed global normalization
schemes are invalid.

We present a novel method for library bias correction (LBC) with a 2D function that simultaneously depends on GC content and gene length. Our approach is unique by modeling not the 
absolute biases but the sample-specific deviations from the data-set wide bias. The computed correction
factors are precise even in the case of partial zero counts. The presented method is useful for correcting 
data sets with subsets of deviating samples as well as for the joined analysis of different data
sets generated with different biases.

     

\noindent \paragraph{Authors:} 
\paragraph{}

{
\centering
\resizebox{\textwidth}{!}{%
\begin{tabular}{|l|l|l|l|l|l|}
\hline
\textbf{first name} & \textbf{last name} & \textbf{email}              & \textbf{country} & \textbf{organization} & \textbf{corresponding?} \\ \hline
Hubert              & Rehrauer           & hubert.rehrauer@fgcz.uzh.ch & Switzerland      & ETH Zurich            & $\checkmark$           \\ \hline
Slavica             & Dimitrieva         &                             & Switzerland      & ETH Zurich            &                         \\ \hline
Ralph               & Schlapbach         &                             & Switzerland      & ETH Zurich            &                         \\ \hline
\end{tabular}
}
}
