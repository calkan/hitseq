\noindent
\large {\bf Genotyping of Inversions and Tandem Duplications} 


\normalsize 


\noindent \paragraph{Keywords:} Next-generation sequencing, Genotyping, Structural variant, Duplication, Inversion

\noindent \paragraph{Abstract:} 

Next Generation Sequencing (NGS) has enabled studying structural genomic variants (SVs)
such as duplications and inversions in large cohorts. SVs have been shown to play important roles in
multiple diseases, including cancer. As costs for NGS continue to decline and variant databases become
ever more complete, the relevance of genotyping also SVs from NGS data increases steadily, which is in
stark contrast to the lack of tools to do so.

We introduce a novel statistical approach, called DIGTYPER (Duplication and Inversion
GenoTYPER), which computes genotype likelihoods for a given inversion or duplication and reports
the maximum likelihood genotype. In contrast to purely coverage-based approaches, DIGTYPER uses
breakpoint-spanning read pairs as well as split alignments for genotyping, enabling typing also of small
events. We tested our approach on simulated and on real data and compared the genotype predictions to
those made by DELLY, which discovers SVs and computes genotypes. DIGTYPER compares favorable
especially for duplications (of all lengths) and for shorter inversions (up to 300 bp). In contrast to DELLY,
our approach can genotype SVs from data bases without having to rediscover them.

Availability: \url{https://bitbucket.org/jana_ebler/digtyper.git}

\noindent \paragraph{Authors:} 

\noindent \paragraph{} 

{
\centering
\resizebox{\textwidth}{!}{%
\begin{tabular}{|l|l|l|l|l|l|}
\hline
\textbf{first name} & \textbf{last name} & \textbf{email}             & \textbf{country} & \textbf{organization}                                                                                   & \textbf{corresponding?} \\ \hline
Jana                & Ebler              &                            & Germany          & Saarland University                                                                                     &                         \\ \hline
Alexander           & Schönhuth          &                            & The Netherlands  & Centrum Wiskunde \& Informatica                                                                         &                         \\ \hline
Tobias              & Marschall          & t.marschall@mpi-inf.mpg.de & Germany          & \begin{tabular}[c]{@{}l@{}}Saarland University and \\ Max Planck Institute for Informatics\end{tabular} & $\checkmark$                        \\ \hline
\end{tabular}
}
}
