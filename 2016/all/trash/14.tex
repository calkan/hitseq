\noindent
\large {\bf Evolution of Structural Variation in Cancer Revealed by Read Clouds} 


\normalsize 


\noindent \paragraph{Keywords:} 

\noindent \paragraph{Abstract:} 


Structural variants, particularly distant translocations, are
difficult to identify despite their fundamental importance in cancer and
other diseases. Because any two genomic loci can be connected through
a genomic rearrangement or translocation, the search space for structural
variation is proportional to the square of the genome size, resulting in a
massive multiple-testing problem for mammalian genomes. Even though
current short-read technologies have very low rates of chimeric molecules
and mismapping to the genome, these types of experimental and computational errors compound to result in high rates of false positives when
searching genome-wide for structural variation. Furthermore, standard sequencing reads derive from short genomic fragments typically only several
hundred base pairs in length, and thus cannot map uniquely to translocation
breakpoints occurring in even moderately long repeat sequences.

The 10X Genomics platform generates barcoded short-reads from
large genomic DNA fragments, which can then be clustered in silico to generate read clouds identifying the original large DNA fragments. We size-selected large (50–100kb) genomic DNA fragments from 7 spatially distinct
tumor samples from a single sarcoma, as well as matched normal tissue, then
applied the 10X platform to generate read clouds.

We have implemented new methods to identify structural variants from
these read cloud data. We use the read cloud barcodes to identify candidate
events where the similarity in barcode patterns between two loci is higher
than expected given the distance between the loci. We then perform breakpoint refinement using the patterns of dropoff in observed long fragment
density at the structural variant breakpoints.

Using this new method, we find structural variants that differ between
sectors of the sarcoma, although most somatic structural variants (and
single-nucleotide variants) are shared across all samples in the tumor. Multiple, independent, ancestral chromothripsis events occurred in our sarcoma
case, totaling hundreds of individual breakpoints shared between sectors.

To better understand these bursts of genome rearrangement, we have
implemented a novel approach using patterns of read clouds to automatically reconstruct the order and orientation of complex structural variants
involving many breakpoints. Furthermore, using the read cloud barcodes,
we are able to identify all reads supporting a structural variant and assemble
the full sequence of many of these complex structural variants (although this
is still dependent on the local sequence complexity). This approach reveals
that many of the complex structural variants involve the rearrangement of
many short (several kb) genomic segments derived from distant locations
on the same chromosome, forming new chromosomes. In the process of creating these neochromosomes, large intervening genomic segments are lost,
resulting in a loss of heterozygosity.

 By harnessing the barcoded sequencing platform, we are
able to phase and assemble complex genomic rearrangements, illuminating larger patterns of genome evolution in cancer. Because the read clouds
derive from long DNA fragments, physical coverage of each breakpoint is
substantially higher than for standard short-read data, resulting in a much
higher signal-to-background. This approach is also able to identify structural
variant breakpoints occurring in repetitive genomic regions, and can actually assemble the nucleotide sequences of these events. Finally, our results
demonstrate that even very large (in this case, over 20 cm in length) tumors
need not show substantial subclonal diversity, and that rather a series of
extreme genomic rearrangements occurred early in tumor development.

\noindent \paragraph{Authors:} 

