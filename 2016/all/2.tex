\noindent
\large {\bf Metagenomic proxy assemblies of single cell genomes} 


\normalsize 


\noindent \paragraph{Keywords:} 
Metagenomics,
Single Cell Genomics,
Sequence Assembly,
Microbial Dark Matter

\noindent \paragraph{Abstract:} 

Over 99\% of the microbial species observed in nature cannot be grown in pure culture,
making it impossible to study them using classical genomic methods. Metagenomics
and single cell genomics are two complimentary approaches to study the microbial dark
matter.

Metagenomics can obtain genome sequences from uncultivated microbes through
direct sequencing of environmental DNA. Each genome’s metagenomic coverage is
constant and depends only on its abundance. A complementary approach to
sequencing DNA of a whole microbial community is single cell genomics. Prior to
sequencing of a single cell, its DNA needs to be amplified. This usually is done by
multiple displacement amplification (MDA), introducing a tremendous coverage bias.
Poorly amplified regions result in extremely low sequencing coverage or physical
sequencing gaps. These parts of the genome cannot be reconstructed in the
subsequent assembly step, and therefore genomic information is lost.

Frequently, single amplified genomes (SAGs) and shotgun metagenomes are generated
from the same environmental sample. We developed a fast, k-mer based recruitment
method to sensitively identify metagenomic “proxy” reads representing the single cell of
interest, using the raw single cell sequencing reads as recruitment seeds. By
assembling metagenomic proxy reads instead of the single cell reads, we circumvent
most challenges of single cell assembly, such as the aforementioned coverage bias and
chimeric MDA products. In a final step, the original single cell reads are used for quality
assessment of the proxy assembly.
On real and simulated data we show that, with sufficient metagenomic coverage,
assembling metagenomic proxy reads instead of single cell reads significantly improves
assembly contiguity while maintaining the original accuracy. By applying our method
iteratively, we span physical sequencing gaps and are able to recover genomic regions
that otherwise would have been lost. However, careful contamination screening is
needed.

We developed kgrep, a new tool that naturally exploits the complementary nature of
single cells and metagenomes to improve de novo assembly of single cell genomes.

\noindent \paragraph{Authors:} 

\paragraph{} 

% Please add the following required packages to your document preamble:
% \usepackage{graphicx}

{
\centering
\resizebox{\textwidth}{!}{%
\begin{tabular}{|l|l|l|l|l|l|l|}
\hline
\textbf{first name} & \textbf{last name} & \textbf{email}                    & \textbf{country} & \textbf{organization}             & \textbf{Web page} & \textbf{corresponding?} \\ \hline
Andreas             & Bremges            & abremges@cebitec.uni-bielefeld.de & Germany          & Bielefeld University              &                   & $\checkmark$                       \\ \hline
Jessica             & Jarett             & jkjarett@lbl.gov                  & United States    & DOE Joint Genome Institute        &                   &                         \\ \hline
Tanja               & Woyke              & twoyke@lbl.gov                    & United States    & DOE Joint Genome Institute        &                   &                         \\ \hline
Alexander           & Sczyrba            & asczyrba@cebitec.uni-bielefeld.de & Germany          & asczyrba@cebitec.uni-bielefeld.de &                   & $\checkmark$                       \\ \hline
\end{tabular}%
}
}



