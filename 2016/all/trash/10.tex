\noindent
\large {\bf Low memory, fast, specific, sensitive, multi-reference sequence classification using Bloom filter maps} 


\normalsize 


\noindent \paragraph{Keywords:} 

\noindent \paragraph{Abstract:} 

Sequence classification is traditionally performed by alignments of sequencing reads onto a 
reference sequence set. Although alignment methodologies have the potential to map the 
location of these reads precisely, this information is not a prerequisite for classification and thus 
perform more computation than is needed. Hash table based methods provide fast access for 
classification but require a large amount of memory. To address these shortcomings, we 
previously proposed an efficient classification method, BioBloom Tools (Chu ​
et al.  ​
2014), that 
uses a low memory, probabilistic set membership query data structure called a Bloom filter.

Using a Bloom filter, elements of a sequence, such as k­mers, are queried to determine 
whether they are or not members of those decomposed from the reference set. The memory 
and time benefits of the data structure have spurned the development of classification tools 
such as  FACs (Stranneheim ​
et. al. ​
 2010) and BioBloom Tools. However, querying for the set of 
origin between multiple reference sets requires the construction and usage of multiple Bloom 
filters leading to $O(n)$ time complexity when querying, where n is the numbers of reference 
sets/filters. Here we present a new Bloom filter based data structure called a Bloom Map that 
can act as an associative array, storing and querying the identifier to the set of origin of a 
specific element in $O(1)$ time .

Conceptually, a Bloom map is a simply a Bloom filter with buckets that are larger than a single 
bit. In our implementation we first construct a normal Bloom filter by hashing our elements and 
set the corresponding buckets in the bit vector to 1. We then interleave rank information into the 
bit vector. Then, we fill an ID array the size of the population count of the filled bloom filter. 
Finally, we hash the elements again, but this time setting elements in the ID array with the 
corresponding IDs of the reference sets according to their rank in the bloom filter. This saves 
memory by effectively reducing the space of each empty bucket to a single bit. To query we 
check the bit vector and then use the rank information to look up the ID array for the identity of 
the queried element in $O(1)$ time. 

We compared our tool against a metagenomic classification tool called Kraken (Wood & 
Salzburg 2014), and its spaced seed counterpart Seed­Kraken (Brinda, ​
et. al. ​ 2015) on the 
NCBI bacterial database. Though our tool is designed for general purpose classification, we 
correctly classified the genus of 97\% reads compared to to Kraken’s 92\% and Seed­Kraken 
95\%, while utilizing less memory (61 GB RSS + 0GB pre­cache) than both Kraken (73GB RSS 
+ 66GB pre­cache) and Seed­Kraken (71GB RSS + 64GB pre­cache) on a read simulated 
dataset. The simulated dataset was constructed using dwgsim and on the genomes used in the 
bacterial database mimicking ~1mil 2x150bp Illumina reads. To investigate specificity we 
introduced 50282 random sequence reads into our simulated read set; Seed­Kraken incorrectly 
assigned a single random read to a genus, but both Kraken and our method managed to not assign any random sequences to a genus. On 8 cores our tool took $<$2 minutes to run on our 
simulated 1mil bp dataset.
 
Hash collisions are dealt with by using logic such as a majority hit rule in addition to 
assigning heavily colliding reference IDs a mutual collision ID. Other features of our 
implementation is the utilization of a recursive rolling hash called ntHash for speed, as well as 
using complementary spaced seeds patterns instead of the traditional use of multiple hash 
functions to improve both sensitivity and specificity. Unlike Kraken our k­mer/seed sizes do not 
affect the memory of our method, which gives BBT the potential to reach a higher specificity by 
using longer seed k-mer/seed sizes. 

Sequence classification to a set of known reference sequences has many applications in 
contamination screening, pathogen detection, metagenomics, and preprocessing for targeted 
assembly from shotgun sequence data. Here we present an efficient low memory alternative to 
hash tables for general purpose, multi­reference sequence classification with broad 
applications, including taxonomic characterization of bio­organisms from metagenomics 
samples.


\noindent \paragraph{Authors:} 

