\noindent
\large {\bf iMapSplice: a lightweight and personalized RNA-seq alignment approach to improve transcriptome profiling} 


\normalsize 


\noindent \paragraph{Keywords:} RNA-seq, alignment, transcriptome profiling

\noindent \paragraph{Abstract:} 


Genomic variants in both coding and non-coding sequences can have unexpected and
functionally important effects on the splicing of gene transcripts. These events, can be measured by
RNA sequencing (RNA-seq), but require the accurate alignment of reads across exon splice junctions.
Existing alignment algorithms that utilize a standard reference genome as a template may have difficulty in
mapping those reads that carry genomic variants. These problems can lead to bias in relative expression
abundance of alternative alleles and the failure to detect splice variants created by splice site mutations.

 To improve RNA-seq read alignments, we have developed a novel lightweight approach called
iMapSplice (Individualized MapSplice) that enables personalized mRNA transcriptional profiling. The
algorithm makes use of personal genomic information and performs an unbiased alignment towards
genome indices carrying both reference and alternative bases. Importantly, this breaks the limit of
dependency on reference genome splice site dinucleotide motifs and enables iMapSplice to discover
personal splice junctions created through splice site mutations. We report comparative analyses by
applying iMapSplice, MapSplice, and STAR on 4 simulated and 68 real human datasets. Besides general
improvements in read alignment and splice junction discovery, iMapSplice greatly alleviates biases in
allelic ratios across the genome and unravels many previously uncharacterized splice junctions created
by mutations at splice sites, with minimal overhead in computation time and storage.

Availability: iMapSplice is implemented as stand alone C++ code, and can be downloaded via URL:
\url{https://github.com/xa6xa6/mps}

\noindent \paragraph{Authors:} 

\noindent \paragraph{} 

% Please add the following required packages to your document preamble:
% \usepackage{graphicx}
{
\centering
\resizebox{\textwidth}{!}{%
\begin{tabular}{|l|l|l|l|l|l|}
\hline
\textbf{first name} & \textbf{last name} & \textbf{email}  & \textbf{country} & \textbf{organization}  & \textbf{corresponding?} \\ \hline
Xinan               & Liu                &                 & United States    & University of Kentucky &                         \\ \hline
James N.            & MacLeod            &                 & United States    & University of Kentucky &                         \\ \hline
Jinze               & Liu                & liuj@cs.uky.edu & United States    & University of Kentucky &        $\checkmark$                 \\ \hline
\end{tabular}%
}
}


