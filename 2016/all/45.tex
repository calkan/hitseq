\noindent
\large {\bf Fast and Accurate Alignment of Single Molecule Maps} 


\normalsize 


\noindent \paragraph{Keywords:} Optical Maps, Alignment, Rmap, Restriction Map

\noindent \paragraph{Abstract:} 

An optical map is an ordered genome-wide high-resolution restriction map that indicates the positions of one or
more short nucleotide sequences. Since optical maps are derived
independently of short sequence reads, they are used in {\it de novo}
genome assembly for validating a draft genome (Nature 2013),
finding structural variation (Nature Methods 2015), and detecting
mis-assembled regions within draft genomes (ISMB 2015). Single
molecule maps, referred to as {\it Rmaps}, is the raw optical mapping
data used to construct the genome-wide optical maps which are
used for aiding in genome assembly. Currently, there exists very few
computational methods to find alignments between the Rmaps - a
task that is the first step in assembling the Rmap data into a genome-wide
 optical maps. and is challenging due to various Rmap specific
errors and their frequency.

 We present {\sc doppelganger}, the first index-based, fully
error-tolerant alignment method for optical mapping data. All
prior alignment methods use dynamic programming, which is
computationally expensive, or have limited error tolerance. We
demonstrate a 20x speedup on the plum genome and demonstrate
on the Ecoli genome the validity of our alignments. Thus,
{\sc doppelganger} is the only non-proprietary method that is capable of
performing pairwise Rmap alignment for large eukaryote organisms
in reasonable time. Lastly, we conclude with other applications of
{\sc doppelganger}, such as the alignment of long reads with high error
rate (e.g. PacBio reads) to a genome-wide optical map.

Availability: The {\sc doppelganger} alignment method is available for
download at \url{https://github.com/mmuggli/doppelganger.}

\noindent \paragraph{Authors:} 

\noindent \paragraph{} 

% Please add the following required packages to your document preamble:
% \usepackage{graphicx}
{
\centering
\resizebox{\textwidth}{!}{%
\begin{tabular}{|l|l|l|l|l|l|}
\hline
\textbf{first name} & \textbf{last name} & \textbf{email}                  & \textbf{country} & \textbf{organization}     & \textbf{corresponding?} \\ \hline
Martin D.           & Muggli             & muggli@cs.colostate.edu         & United States    & Colorado State University & $\checkmark$                        \\ \hline
Simon J.            & Puglisi            &                                 & Finland          & University of Helsinki    &                         \\ \hline
Christina           & Boucher            & Christina.Boucher@colostate.edu & United States    & Colorado State University &  $\checkmark$                       \\ \hline
\end{tabular}%
}
}
