\noindent
\large {\bf VarMatch: A fast, parallel, and memory-efficient method for the variant matching problem} 


\normalsize 


\noindent \paragraph{Keywords:} 

\noindent \paragraph{Abstract:} 

Small variant ($\leq$ 30bp) calling is widely used in medical and genetic research to identify how genome
mutations are related to phenotypes of interest. Variant matching is the problem of comparing different
sets of variant calls, to determine the variants that are in common between the sets or unique to each
set. Variant matching can be done to (1) compare the performance of different tools with respect to each
other or with respect to a ground truth. (2) extract high-confidence variants for an individual by taking
the intersection of calls from multiple callers, and (3) find variants that are shared or unique across
different individuals.

A set of small variants is typically represented as a collection of VCF entries, where each entry contains
a position of the reference genome and the alternate diploid allele (e.g. sequence) in the donor. The most
straightforward variant matching algorithm is to directly match identical VCF entries. However, it can
fail to match two different VCF entries that nevertheless result in the same diploid donor genome.
Normalization and decomposition [1–3] have been used to alleviate these problems, however, there are
still alternate representations for the same variant that are not matched [4]. An alternate approach is
to formulate and solve an appropriate optimization problem that finds, roughly speaking, the largest
number of matches [4]. This method can detect equivalent variants unmatched by heuristic algorithms,
but still suffers from large memory usage.

An additional limitation is that these approaches can only support maximizing the number of total
matched VCF entries. However, this is sensitive to whether a tool represents complex variants as a single
entry or as multiple, decomposed, entries. A more representation-invariable optimization criteria would
be to maximize the number of matched nucleotides. In other cases, such as comparing multiple callers
to a ground truth set, it is desirable to instead maximize the total number of matched entries from the
ground truth set only.


To address these problems, we introduce a new algorithm VarMatch. VarMatch is an exact algorithm
for variant matching that is guaranteed to find matching variants under a wide variety of optimization
criteria. VarMatch employs a provably optimal divide and conquer strategy to partition the set of variants
into disjoint subproblems. Because each subproblem is typically very small, we can use an exact dynamic
programming algorithm similar to [4] (for the maximum number of matches optimization criteria) or even
brute force (for other criteria) to solve each subproblem. While our algorithm has exponential running
time in the worst case, we demonstrate it performs very fast in practice and uses an order of magnitude
less memory than [4]. This can be crucial for applications in medical settings, where the software may
be run on embedded processors or portable devices. VarMatch is also a parallel algorithm that scales
over multiple processors and/or threads. Additionally, our divide and conquer strategy makes it easy to
support any optimization criteria for doing matches, since even a brute force implementation is practical
for the small subproblems. We have implemented several scoring functions in VarMatch: (1) maximize
the total number of matched entries, (2) maximize the number of matched entries from one of the call
sets, and (3) maximize the total number of matched bases. VarMatch is implemented as a user-friendly
software package that will be available on GitHub if accepted.

We consider the variant matching problem, roughly defined as follows: given a pair of variant sets V, W ,
find subsets V ∈ W and W ∈ W such that applying V and W results in the same diploid sequence, and
f (V, W ) is maximized. The function f can be almost any computable function, with the most natural
one f (V, W ) = |V | + |W |. If we consider a reference genome interval without any variants, we can split
the input variants into those to the left and to the right of the interval. Our main theoretical result states
that, given a sufficiently long and non-repetitive interval, the solution to the variant matching problem
on V, W is equivalent to the union of the solutions to the problem on V left , W left and V right , W right .
This theorem is significant for two reasons. First, it leads to an exact, parallel, fast and low-memory
divide-and-conquer algorithm that partitions the large problem into smaller subproblems which can be
solved with a brute-force algorithm. Second, it allows the use of any reasonable optimization criteria,
which other algorithms do not allow.

Table 1 illustrates evaluation of VarMatch on two published real data sets [2] with single thread, com-
paring the accuracy, memory usage(RAM) and running time of VarMatch to the normalization approach
(based on Vt [1] followed by direct matching) and to RTG Tools [4]. For dataset CHM1 (Table 1a), we
take variant call sets on the same sequencing data of the CHM1hTERT cell line. Variants were called
separately by FreeBayes and HaplotypeCaller of GATK. For dataset NA12878 (Table 1b), we take vari-
ant call sets by Platypus and UnifiedGenotyper of GATK on NA12878 cell line. Both RTG Tools and
VarMatch match more VCF entries than Vt at the cost of more resources, but VarMatch uses less running
time and an order of magnitude less memory than RTG Tools.

Figure 1a shows the effectiveness of our partitioning approach on dataset CHM1, VarMatch partitions
6,438,208 initial small variants into 3,272,206 subproblems, 99.9\% of which have less than 9 variants in
them. Figure 1b shows that on dataset CHM1 VarMatch scales with multiple threads, while with more
threads I/O becomes the bottleneck(∼ 200 seconds).

\begin{table}[htb]
        \begin{tabular}{} 
\end{table}

\noindent \paragraph{References:} 
\noindent \paragraph{} 
\noindent 1. Tan, A., Abecasis, G. R., and Kang, H. M. Bioinformatics , btv112 (2015).\\
2. Li, H. Bioinformatics 30(20), 2841–2851 (2014).\\
3. Zook, J. M., Chapman, B., Wang, J., Mittelman, D., Hofmann, O., Hide, W., and Salit, M. Nature biotechnology
32, 246–251 (2014).\\
4. Cleary, J. G., Braithwaite, R., Gaastra, K., Hilbush, B. S., Inglis, S., Irvine, S. A., Jackson, A., Littin, R.,
Rathod, M., Ware, D., et al. bioRxiv , 023754 (2015).


\noindent \paragraph{Authors:} 

