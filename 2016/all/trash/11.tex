\noindent
\large {\bf From genomes to phenotypes: Traitar, the microbial trait analyzer} 


\normalsize 


\noindent \paragraph{Keywords:} 

\noindent \paragraph{Abstract:} 


The number of sequenced genomes is growing exponentially, profoundly shifting the
bottleneck from data generation to genome interpretation. Traits are often used to
characterize and distinguish bacteria, and are likely a driving factor in microbial community
composition, yet little is known about the traits of most microbes. We present Traitar, the
microbial trait analyzer, a fully automated software package for deriving phenotypes from the
genome sequence. Traitar accurately predicts 67 traits related to growth, oxygen
requirement, morphology, carbon source utilization, antibiotic susceptibility, amino acid
degradation, proteolysis, carboxylic acid use and enzymatic activity.

Traitar uses L1-regularized L2-loss support vector machines for phenotype assignments,
trained on protein family annotations of a large number of characterized bacterial species, as
well as on their ancestral protein family gains and losses. We demonstrate that Traitar can
reliably phenotype bacteria even based on incomplete single-cell genomes and simulated
draft genomes. We furthermore showcase its application by characterizing two novel
Clostridiales species based on genomes recovered from the metagenomes of commercial
biogas reactors, verifying and complementing a manual metabolic reconstruction.

Traitar enables microbiologists to quickly characterize the rapidly increasing number of
bacterial genomes. It could lead to models of microbial interactions in a natural environment
and inference of the conditions required to grow microbes in pure culture. Our phenotype
prediction framework offers a path to understanding the variation in microbiomes. Traitar is
available at \url{https://github.com/hzi-bifo/traitar}.

\noindent \paragraph{Authors:} 

