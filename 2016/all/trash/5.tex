\noindent
\large {\bf Co-Expression of Long non-coding RNAs with Epigenetically regulated genes in TCGA Glioma subtypes} 


\normalsize 


\noindent \paragraph{Keywords:} 

\noindent \paragraph{Abstract:} 

 In recent years RNA-seq deep sequencing technology has emerged as a
revolutionary tool to precisely measure transcriptome profiling in eukaryotic genomes. Beyond
protein coding RNAs, long non-coding RNAs (lncRNAs) have become recognized as a gene
regulators as well as prognostic markers in cancer. In this study, we initiated an in-silico analysis
of co-expression of lncRNAs with epigenetically regulated genes (EReg) in TCGA Glioblastoma
multiform (GBM) and Lower Grade Glioma (LGG) RNA-seq data.

Open-source RNA-seq data set which is manufactured with Illumina HiSeq platform from
TCGA GBM and LGG cohort were integrated to capture highly correlated bio-molecules, in our
case, lncRNAs and ERegs. A set of 12382 differentially regulated lncRNAs transcripts were
identified across various cancers including GBM & LGG samples (372) were derived from
Chinnaiyan et.al identified by the Tuxedo suite (i.e, Tophat, Cufflink) which perform many aspects
of complete RNA-seq analysis in ab initio assembly mode. A set of 809 EReg transcripts were
obtained from Cecceralli et al. which categorizes 7 distinct glioma subtypes in IDHmutant
(codal=69,G-CIMP-high=104,G-CIMP-low=8)
and
IDHwildtype
(Classic-like=54,LGm6-
GBM=12,Mesenchymal-like=69,PA-like=15) by unsupervised clustering of Illumina methylation
27k and 450k array probes. The expression estimates of these EReg transcripts were generated
by Mapsplice/RSEM workflow constructed by broad Institute, were downloaded from GDAC.
Expression estimates of lncRNAs were in FPKM and ERegs were in estimated transcript fraction,
as these two measures were generated by two different algorithms/workflow, so they were made
compatible by converting to transcripts per million (TPM). Following data processing and QC on
this integrated data, 315 samples and 12991 (lncRNA=12195 and EReg=796 transcripts)
molecules were carried forward for analysis with Weighted Correlation network analysis.
Following detection of networks which consists of lncRNAs and ERegs, association of these co-
expression networks to glioma subtypes were analyzed with Anova. EReg genes from
significantly associated modules were further analyzed by Ingenuity’s IPA to delineate biological
association as majority of lncRNAs have unknown functions, so “guilt by association” mechanism
was used for retrieving functional relevance to these lncRNAs by EReg genes.

There were 27 lncRNA-EReg gene modules were detected. Among these, 2 modules
were significantly associated 2 glioma subtypes (IDHWt = PA-Like and IDHWt = LGm6-GBM) at
pvalue $<$ 0.05. After multiple testing corrections, both of these modules remain as significant at
FDR level $<$ 0.05. EReg genes which were extracted from module associated with LGm6-GBM
are working together in cell-To-cell Signaling and Interaction, cellular Growth and Proliferation
while EReg genes associated with PA-Like glioma subtype were working together in cell cycle,
cellular development, cellular growth and proliferation biological functions. So it can be assumed
that lncRNA transcripts which were co-expressed with ERegs transcripts from above mentioned
modules will participate in these cellular functions.

 This study demonstrates the application of existing bioinformatics algorithms to
analyze open source RNA-seq data to capture gene-lncRNA association in respect to glioma
subtypes.

\noindent \paragraph{Authors:} 

