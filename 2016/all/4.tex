\noindent
\large {\bf Resource-efficient Assembly of Large Genomes with Bloom Filter ABySS} 


\normalsize 


\noindent \paragraph{Keywords:} 

\noindent \paragraph{Abstract:} 
Since the introduction of the de Bruijn graph assembly approach by Pevzner
et al. in 2001, de Bruijn graph assemblers have become the dominant
method for de novo assembly of large genomes. Nonetheless, assembling
large genomes remains a challenging task. For instance, the estimated mem-
ory requirements for a human genome assembly with the ALLPATHS-LG
assembler is 512GB of RAM. While distributed de Bruijn graph assemblers
such as ABySS, Ray, and PASHA eliminate the requirement for a single
large-memory machine by distributing the de Bruijn graph across multiple
cluster nodes, these assemblers still require a computing cluster with a large
amount of aggregate memory and a high-speed network fabric. While assem-
blers typically represent the de Bruijn graph as a hash table of k-mers, the
Minia assembler (Chikhi et al., 2012) introduced a more compact probabilis-
tic representation using a Bloom filter, which reduces the memory require-
ment by orders of magnitude and renders large genome assemblies feasible
on a single commodity machine.

Here we present two fundamental improvements to the ABySS assembler
that reduce the memory and running time for large genome assemblies.
First, as in Minia, we have reduced memory requirements by an order of
magnitude through the use of a Bloom filter de Bruijn graph. While Minia
1is a standalone unitig assembler, our new Bloom filter assembler is inte-
grated with the existing ABySS pipeline, including downstream stages for
contig building, mate pair scaffolding, and long read scaffolding. Second, we
have reduced assembly time through the use of a specialized hash function
called ``ntHash''. In our application, ntHash achieves runtimes that are or-
ders of magnitude faster than standard hash functions through the use of
a constant-time sliding window calculation, where the hash value of each
k-mer is computed from the hash value of the k-mer that preceeds it. On
a single 32-core machine with 120GB RAM, the new Bloom filter version of
ABySS is able to assemble a modern 76X human dataset (SRA:ERR309932)
and scaffold with MPET data (SRA:ERR262997) with an NG50 of 1.7 Mbp,
wallclock time of 46 hours, and a peak memory usage of 102GB RAM.

While many implementations of de Bruijn graph assemblers are available,
de novo assemblies of large genomes such as Homo sapiens still require
heavy computational resources. Here we have demonstrated improvements
to ABySS with respect to both memory usage and running time that signif-
icantly reduce the cost of assembling large genomes.


\noindent \paragraph{Authors:} 

